\title{Psychological Science Submission Letter}
%
% See http://texblog.org/2013/11/11/latexs-alternative-letter-class-newlfm/
% and http://www.ctan.org/tex-archive/macros/latex/contrib/newlfm
% for more information.
%
\documentclass[10pt,stdletter,orderfromtodate,sigleft]{newlfm}
\usepackage{blindtext, xfrac, animate, hyperref, pxfonts}

  \setlength{\voffset}{0in}

\newlfmP{dateskipbefore=0pt}
\newlfmP{sigsize=20pt}
\newlfmP{sigskipbefore=10pt}
 
\newlfmP{Headlinewd=0pt,Footlinewd=0pt}
 
\namefrom{Jeremy R. Manning, Ph.D.}
\addrfrom{
	Dartmouth College\\
    Department of Psychological \& Brain Sciences\\
    HB 6207 Moore Hall\\
	Hanover, NH  03755}
 
\addrto{}
\dateset{\today}
 
\greetto{To the editors of \textit{Psychological Science}:}
 
\closeline{Sincerely,}

\begin{document}
\begin{newlfm}
We have enclosed our manuscript entitled \textit{Feature-based and
  location-based volitional covert attention affect memory at different timescales}, which we wish
to submit for publication as a Research Article in
\textit{Psychonomic Bulletin \& Review}.

Our manuscript reports two experiments that examine how feature-based
and location-based attention affect recognition memory.  We asked our
participants to view a series of composite (blended face-scene) image
pairs.  We instructed the participants to focus their attention to a
specific image category (face or scene) and location (left or right)
without moving their eyes from a central fixation point.  The two
experiments differed in how often we changed the attention cue.  We
found that the memory consequences of location-based attention were
measurable after just a single attention cue instruction
(5.5~seconds), and persisted for at least 2~minutes.  Our data suggest
that location-based attention enhances memory encoding at the attended
location.  By contrast, feature-based attention must be sustained for
an extended interval to substantially affect memory.  Our data suggest
that feature-based attention suppresses the encoding into memory of
unattended stimulus features.

The impact of attention on memory is of central importance to a wide
variety of scientifically and socially current issues.  For example,
the interplay between attention and memory affects how we remember our
ongoing experiences in everyday life, how students learn in the
classroom (and which teaching strategies might be most effective), how
(or whether) we recognize a face we notice in a crowd and
happen to encounter later, and so on.  These issues also relate to
attention disorders (e.g., ADD, ADHD) and memory disorders
(e.g., age-related memory impairment, Alzheimer's, etc.).  Discovering
the mechanisms that underlie how our experiences are encoded into
memories can lead to new tools for identifying when those mechanisms
are not operating within their expected parameters.

We have suggested the following scientists as expert reviewers: Marvin
Chun (Marvin.Chun@Yale.edu), Brice Kuhl (bkuhl@uoregon.edu), Joel Voss (Joel-Voss@Northwestern.edu),
Edward Vogel (Ed\-Vogel@UChi\-ca\-go.edu), and Judith Fan
(jefan@ucsd.edu).  We have also published and documented all of our
experimental stimuli, code, and data, along with the code used to
generate the figures and analyses we report in our manuscript.  These
may be found at
\href{https://github.com/ContextLab/attention-memory-task}{\texttt{https://github.com/ContextLab/attention-memory-task}}.

Thank you for considering our manuscript for publication in
Psychonomic Bulletin \& Review.

\end{newlfm}
\end{document}


